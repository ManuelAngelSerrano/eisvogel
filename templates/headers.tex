% Enable emoji handling via LuaLaTeX
\usepackage{fontspec}
\usepackage{luacode} % Needed for direct Lua code execution in LaTeX

% Add the emoji fallback mechanism using LuaLaTeX
\directlua{
    luaotfload.add_fallback("emojifallback", {
        "Noto Color Emoji:mode=harf;script=DFLT;"
    })
}

% Set the main and sans-serif fonts with emoji fallback
\setmainfont{Source Sans Pro}[RawFeature={fallback=emojifallback}]
\setsansfont{Source Sans Pro}[RawFeature={fallback=emojifallback}]
\setmonofont{Source Sans Pro}[RawFeature={fallback=emojifallback}]

% Optional: Load emoji-specific package if you plan to use emoji characters via name
\usepackage{emo}

% Esto para poner lineas de separación en cabecera y pie de página
\usepackage{fancyhdr}% http://ctan.org/pkg/fancyhdr
\pagestyle{fancy}% Change page style to fancy
\fancyhf{}% Clear header/footer
%\fancyhead[C]{Header}
%\fancyfoot[C]{Footer}
\fancyfoot[R]{\thepage}
\renewcommand{\headrulewidth}{0.4pt}% Default \headrulewidth is 0.4pt
\renewcommand{\footrulewidth}{0.4pt}% Default \footrulewidth is 0pt
